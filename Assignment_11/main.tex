%%%%%%%%%%%%%%%%%%%%%%%%%%%%%%%%%%%%%%%%%%%%%%%%%%%%%%%%%%%%%%%
%
% Welcome to Overleaf --- just edit your LaTeX on the left,
% and we'll compile it for you on the right. If you open the
% 'Share' menu, you can invite other users to edit at the same
% time. See www.overleaf.com/learn for more info. Enjoy!
%
%%%%%%%%%%%%%%%%%%%%%%%%%%%%%%%%%%%%%%%%%%%%%%%%%%%%%%%%%%%%%%%


% Inbuilt themes in beamer
\documentclass[15pt]{beamer}


% Theme choice:
\usetheme{Warsaw}
\usecolortheme{rose}

% Title page details: 
\title{Assignment-11} 
\author{Shivanshu  Ai21btech11027}
\date{june 14, 2022}
\begin{document}
\begin{frame}
    \titlepage 
\end{frame}

\begin{frame}{Outline}
    \tableofcontents
\end{frame}

    \section{Question}
    \begin{frame}{Question}
        \textbf{Papoulis book exercise 11}\\
        \large \noindent Q-10 Show that if\\
        \begin{center}
             $E\{x_n , x_k\} = \sigma^2\delta[n - k] \hspace{5mm} X(\omega) = \sum_{n=-\infty}^\infty x_ne^{-jn\omega T} $
        \end{center}
        and $E\{x_n\} = 0$, then $E\{X(\omega)\} = 0$ and \\
        $E\{X(u)X^*(v)\} = \sum_{n=-\infty}^\infty \sigma^2_n e^{-jn(u-v)T}$
    \end{frame}
    \section{Solution}
    \begin{frame}{Solution}
        \begin{align}
            X(u) &= \sum_{n=-\infty}^\infty x_n e^{-jnuT} \\
            X^*(u) &= \sum_{k=-\infty}^\infty x_k e^{-jkuT} \\
            X(u)X^*(u) &= \sum_{n=-\infty}^\infty \sum_{k=-\infty}^\infty ( x_n e^{-jnuT} x_k e^{-jkuT} ) \\
            &= \sum_{n=-\infty}^\infty \sum_{k=-\infty}^\infty (x_n x_k)(e^{-j(nu-kv)T}) \\
            E(X(u)X^*(u)) &=  \sum_{n=-\infty}^\infty \sum_{k=-\infty}^\infty (e^{-j(nu-kv)T}) E(x_n x_k)
        \end{align}
    \end{frame}
    \begin{frame}{Solution}
        \begin{align}
            &= \sum_{n=-\infty}^\infty \sum_{k=-\infty}^\infty \sigma_n^2\delta(n - k) (e^{-j(nu - kv)T}) \\
            &= \sum_{n=-\infty}^\infty \sigma_n^2 e^{-jn(u - v)T}
        \end{align}
    \end{frame}
\end{document}
